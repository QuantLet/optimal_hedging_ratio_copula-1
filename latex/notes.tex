%%
%% $Id: article.tex,v 1.1 2008/09/20 10:19:28 natalie Exp $
%% $Source: /Users/natalie/cvs/tex/templates/article.tex,v $
%% $Date: 2008/09/20 10:19:28 $
%% $Revision: 1.1 $
%%

%\documentclass[a4paper,11pt,BCOR1cm,DIV11,headinclude]{scrbook}
% bei 12pt ist DIV 12 default, bei 11pt ist es DIV 10
% Textbereiche 
% DIV 10: 147*207.9mm, DIV 11: 152.73*216mm, DIV 12:157.50*222.75
% DIV 13: 161.54*228.46mm, DIV 14: 165*233.36mm

\def\deftitle{Notes on hedging cryptos with spectral risk measures}
% \def\defauthor{N.\ Packham}
% \def\defauthor{nat}
\def\defauthor{}

%% option: largefont
\documentclass[square]{article} %
%% options: vscreen, garamond, wnotes, savespace
\usepackage[vscreen]{nat}
\usepackage[longnamesfirst]{natbib}
\bibpunct{(}{)}{;}{a}{,}{,}
\usepackage{amsfonts,amssymb,amsthm} %
\usepackage[tbtags]{amsmath} %
% \usepackage{fullpage}

\usepackage{graphicx,color}
\graphicspath{{./pics/}}
\definecolor{BrickRed}{rgb}{.625,.25,.25}
\providecommand{\red}[1]{\textcolor{BrickRed}{#1}}
\definecolor{markergreen}{rgb}{0.6, 1.0, 0}
\definecolor{darkgreen}{rgb}{0, .5, 0}
\definecolor{darkred}{rgb}{.7,0,0}
\providecommand{\marker}[1]{\fcolorbox{markergreen}{markergreen}{{#1}}}
\providecommand{\natp}[1]{\textcolor{darkred}{#1}}
\theoremstyle{plain}
\newtheorem{theorem}{Theorem}%[section]
\newtheorem{proposition}[theorem]{Proposition}
\newtheorem{corollary}[theorem]{Corollary} %%
\newtheorem{lemma}[theorem]{Lemma} %%
\theoremstyle{definition} %%
\newtheorem{definition}{Definition}
\newtheorem{remark}[theorem]{Remark}
\newtheorem{remarks}{Remarks}
\newtheorem{condition}[theorem]{Condition}
\newtheorem{example}[theorem]{Example}
\newtheorem{assumption}{Assumption}

\usepackage[makestderr]{pythontex}
\begin{pycode}
import numpy as np
from scipy import stats
import statsmodels.api as sm
import pandas as pd
import matplotlib.pyplot as plt
np.random.seed(87654)
\end{pycode}


%%
%% $Id: Definitions.tex,v 1.6 2008/07/26 14:55:50 natalie Exp $
%% $Source: /Users/natalie/cvs/tex/dynamics/Definitions.tex,v $
%% $Date: 2008/07/26 14:55:50 $
%% $Revision: 1.6 $
%%

%\usepackage{mathrsfs}

%% GENERAL DEFINITIONS
\unitlength1cm

%% COMMAND DEFINITIONS
\newcommand{\E}{{\mathbb{E}}}
%%\renewcommand{\E}{{\mathds E}}
%%\renewcommand{\E}{{\varmathbb{E}}}
%%\renewcommand{\E}{{\mathrm{I\!E}}}
\providecommand{\R}{{\mathbb{R}}}
\newcommand{\T}{{\mathbb{T}}}
\newcommand{\Fb}{{\mathbb{F}}}
\newcommand{\Eqn}{{\mathbb{E}}_{{\bf Q}_N}}
\newcommand{\Eq}{{\mathbb{E}}_{{\bf Q}}}
\newcommand{\Eqm}{{\mathbb{E}}_{{\bf Q}_M}}
\newcommand{\EqT}{{\mathbb{E}}_{{\bf Q}_T}}
\newcommand{\EqTz}{{\mathbb{E}}_{{\bf Q}_{T_2}}}
\newcommand{\EqTe}{{\mathbb{E}}_{{\bf Q}_{T_1}}}
\newcommand{\EqSe}{{\mathbb{E}}_{{\bf Q}_{S^1}}}
\newcommand{\EqSz}{{\mathbb{E}}_{{\bf Q}_{S^2}}}
\newcommand{\p}{{\bf P}}
%%\renewcommand{\p}{{\mathds{P}}}
%%\renewcommand{\p}{{\varmathbb{P}}}
%%\renewcommand{\p}{{\mathrm{I\!P}}}
\newcommand{\pas}{\text{{\bf P}--a.s.}}
\newcommand{\paa}{\text{{\bf P}--a.a.}}
\newcommand{\qas}{\text{{\bf Q}--a.s.}}
\newcommand{\e}{{\bf e}}
\newcommand{\q}{{\bf Q}}
\newcommand{\qn}{{\bf Q}_N}
\newcommand{\qm}{{\bf Q}_M}
\newcommand{\qT}{{\bf Q}_T}
\newcommand{\qTz}{{\bf Q}_{T_2}}
\newcommand{\qTe}{{\bf Q}_{T_1}}
\newcommand{\qS}{{\bf Q}_S}
\newcommand{\qSe}{{\bf Q}_{S^1}}
\newcommand{\qSz}{{\bf Q}_{S^2}}
\newcommand{\F}{{\cal F}}
\newcommand{\G}{{\cal G}}
\newcommand{\A}{{\cal A}}
\newcommand{\Hc}{{\cal H}}
\newcommand{\dP}{{\rm d}{\bf P}}
\newcommand{\du}{{\rm d}u}
%%\newcommand{\dt}{{\rm d}t}
\newcommand{\dd}{{\rm d}}
\newcommand{\df}{{\rm \bf DF}}
\providecommand{\N}{{\mathbb N}}
\providecommand{\Ncdf}{{\rm N}}
%\renewcommand{\Ncdf}{{\Phi}}
\newcommand{\n}{{\rm n}}
\newcommand{\emb}{\bf \em}
\newcommand{\1}{\textbf{1}}
\newcommand{\qs}{{\q_{\rm Swap}}}
\newcommand{\fx}{{\rm fx}}
\newcommand{\V}{{\rm Var}}
%\newcommand{\C}{{\bf C}}
\newcommand{\Om}{{\Omega}}
\providecommand{\limn}{\ensuremath{\lim_{n\rightarrow\infty}}}
\providecommand{\qv}[2]{\ensuremath{\langle #1,#1\rangle_{#2}}}

%% ENVIRONMENT DEFINITIONS
%\newtheorem{prop}{Proposition}[section]
%\newtheorem{theo}{Theorem}[section]
%\newtheorem{lem}{Lemma}[section]
%\newtheorem{ass}{Assumption}[section]
%\newtheorem{cor}{Corollary}[section]
%\newtheorem{aufg}{Exercise}[section]
%\newtheorem{defi}{Definition}[section]

\ifx\prop\undefined
\newtheorem{prop}{Proposition}[section]
\fi
\newtheorem{theo}[prop]{Theorem}
\newtheorem{lem}[prop]{Lemma}
\newtheorem{cor}[prop]{Corollary}
\newtheorem{defi}[prop]{Definition}

%% enumeration in lists
\providecommand{\labelenumi}{{\rm (\roman{enumi})}}
   %\setlength{\topsep}{0cm}
    \setlength{\labelsep}{0.3cm}
    %\setlength{\itemindent}{0cm}
   \setlength{\leftmargin}{10cm}
    \setlength{\labelwidth}{5cm}

\providecommand{\cadlag}{c\`adl\`ag }
\providecommand{\cadlagns}{c\`adl\`ag}
\providecommand{\caglad}{c\`agl\`ad }
\providecommand{\cad}{c\`ad}
\providecommand{\cag}{c\`ag}
\providecommand{\levy}{L\'evy\ }
\providecommand{\levyns}{L\'evy}
\providecommand{\levyito}{L\'evy-It\^o\ } 
\providecommand{\levykhinchin}{L\'evy-Khinchin\ }
\providecommand{\D}{\ensuremath{D(\R_+,\R)}}
\providecommand{\Dsig}{\ensuremath{D(\R_+, \R_+\setminus\{0\}})}
\providecommand{\Dd}{\ensuremath{D(\R_+,\R^d)}}
\providecommand{\C}{\ensuremath{C(\R_+,\R)}}
\providecommand{\Cd}{\ensuremath{C(\R_+,\R^d)}}
\providecommand{\rpos}{\ensuremath{{[0,\infty)}}}

\def\Z{{\mathbb Z}}
%\def\N{{\mathbb N}}
%\def\R{{\mathbb R}}
%\def\C{{\mathbb C}}
%\def\H{{\mathbb H}}
\def\P{{\mathbb P}}
\def\Q{{\mathbb Q}}
%\def\E{{\mathbb E}}
\def\I{{\mathbb I}}
%\def\T{{\mathbb T}}
%\def\F{{\mathbb F}}
\def\M{{\mathbb M}}
%\def\Hc{{\mathcal H}}
\def\Mc{{\mathcal M}}
\def\filtration#1{{\ensuremath\mathcal{#1}}}
%\def\filt{{\mathcal F}}
\def\tp{\tilde{\p}}
\providecommand{\vec}[1]{\ensuremath{\bm #1}}
\providecommand{\vecb}[1]{\ensuremath{\bm #1}}
\providecommand{\abs}[1]{\ensuremath{\lvert#1\rvert}}
\providecommand{\norm}[1]{\ensuremath{\lVert#1\rVert}}
\providecommand{\var}{\ensuremath{\text{Var}}}
\providecommand{\cov}{\ensuremath{\text{Cov}}}
\providecommand{\borel}[0]{\ensuremath{\mathcal{B}}}
\providecommand{\intinf}[0]{\ensuremath{\int_{-\infty}^\infty}}
\providecommand{\intpos}[0]{\ensuremath{\int_0^\infty}}
\providecommand{\intneg}[0]{\ensuremath{\int_{-\infty}^0}}
\providecommand{\todo}[1]{\footnote{#1}}
\providecommand{\dynkin}[0]{\ensuremath{\mathcal D}}
\providecommand{\ce}[2]{\ensuremath{\E(#1|\filtration{#2})}}
\providecommand{\inv}[1]{\ensuremath{#1^{(-1)}}}
\providecommand{\os}[2]{\ensuremath{#1^{(#2)}}}
\providecommand{\pos}[2]{\ensuremath{h_{#1}(#2)}}
%\providecommand{\poslong}[2]{\ensuremath{h(#1, #2)}}
\providecommand{\poslong}[3]{\ensuremath{h_{#1, #2}(#3)}}

%% Class of finite variation processes
\providecommand{\classfv}{\ensuremath{\mathscr V}}
\providecommand{\classv}{\ensuremath{\mathscr V}}
%% Stochastic integral operator
\providecommand{\stint}{\ensuremath{\cdotp}}
\providecommand{\classh}{\ensuremath{\mathscr H^2}}
\providecommand{\classhloc}{\ensuremath{\mathscr H^2_{\rm loc}}}
\providecommand{\classm}{\ensuremath{\mathscr M}}
\providecommand{\classmloc}{\ensuremath{\mathscr M_{\rm loc}}}
\providecommand{\classl}{\ensuremath{L^2}}
\providecommand{\classlloc}{\ensuremath{L^2_{\rm loc}}}
\providecommand{\classa}{\ensuremath{\mathscr A}}
\providecommand{\classaloc}{\ensuremath{\mathscr A_{\rm loc}}}
\providecommand{\classalocpos}{\ensuremath{\mathscr A_{\rm loc}^+}}
\providecommand{\classp}{\ensuremath{\mathscr P}}
\providecommand{\classo}{\ensuremath{\mathscr O}}
\providecommand{\classs}{\ensuremath{\mathscr S}}
\providecommand{\classsp}{\ensuremath{\mathscr S_p}}
\providecommand{\nullset}{\ensuremath{\mathscr N}}

\providecommand{\ito}{It\^o }
\providecommand{\itos}{It\^o's\, }

\providecommand{\variation}[2]{\ensuremath{\rm V_{#1}(#2)}}
\renewcommand{\H}{\ensuremath{\mathcal H}}
%% CPO distribution
\providecommand{\cpo}{\ensuremath{{\rm CPO}}}
\providecommand{\Fsigma}{\ensuremath{\mathcal \F_\infty^\sigma}}
\providecommand{\sigd}{\ensuremath{\mathscr D}}

%% Credit spreads
\providecommand{\s}{{\bf s}}
\providecommand{\classu}{\ensuremath{\mathscr U}}

\providecommand{\sX}{\ensuremath{\mathcal X}}
\providecommand{\sY}{\ensuremath{\mathcal Y}}
\providecommand{\dx}{\ensuremath{\frac{\partial}{\partial x}}} %%
\providecommand{\dt}{\ensuremath{\frac{\partial}{\partial t}}} %%
\providecommand{\dy}{\ensuremath{\frac{\partial}{\partial y}}} %%
\newcommand{\argmax}{\operatornamewithlimits{argmax}}
\newcommand{\argmin}{\operatornamewithlimits{argmin}}

\sloppy
\begin{document}
\setlength{\boxlength}{0.95\textwidth} %
\title{\large{\bf\deftitle}} %
\author{{\normalsize\bf\defauthor}}%
\thispagestyle{empty}
\addtocounter{page}{1}
\maketitle
\begin{abstract}
  We investigate different methods of hedging cryptocurrencies with
  Bitcoin futures. A useful generalisation of variance-based hedging
  uses spectral risk measures and copulas. 
\end{abstract}
% \keywords{keywords here} %%
% \jel{jel here} %%
\vspace{.5cm}
\def\contentsname{Contents}
\tableofcontents
%%
\vspace{.5cm}
%\clearpage

\section{Optimal hedge ratio}
\label{sec:optimal-hedge-ratio}

Following \citep{Barbi2014}, we consider the problem of the optimal
hedge ratios by extending commmonly known minimum variance hedge ratio
to more general risk measures and dependence structures.\medskip\\
Hedge portfolio: $R_t^h = R_t^S - h R_t^F$, involving returns of spot
and future contract and where $h$ is the hedge ratio\\
Optimal hedge ratio: $h^\ast = \argmin_h \rho_\phi(s,h)$, for given
confidence level $1-s$ (if applicable, e.g.\ in the case of VaR, ES),
where $\rho_\phi$ is a spectral risk measure with weighting function
$\phi$ (see below). \\ 
Corollary 2.1 of \citep{Barbi2014}, corrected: Let $R^S$ and $R^F$ be
two real-valued random variables on the same probability space
$(\Omega, \mathcal A, \p)$ with corresponding absolutely continuous
copula $C^t_{R^S, R^F}(w,\lambda)$ and continuous marginals $F_{R^S}$
and $F_{R^F}$. Then, the $s$-quantile of $R^h$ solves the following:
\begin{equation*}
  F_{R^h}(r^h) = 1- \int^1_0 D_1 C_{R^S, R^F}
  \left\{ w, F_{R^F} \left[ \frac{F^{-1}_{R^S}(w)-r^h}{h} \right]
  \right\}dw.  
\end{equation*}
[..]\\
Here $D_1 C(u,v)=\displaystyle \frac{\partial}{\partial u} C(u,v)$,
which can be shown to fulfil \citep{Cherubini2011}
\begin{equation*}
  D_1 C_{X,Y}(F_X(x), F_Y(y)) = \p(Y\leq y|X=x).
\end{equation*}

\section{Spectral risk measures}
\label{sec:spectr-risk-meas}

Spectral risk measure \citep{Acerbi2002,Cotter2006}:
\begin{equation*}
\rho_\phi = -\int_0^1 \phi(p)\, q_p\, \dd p,
\end{equation*}
where $q_p$ is the $p$-quantile of the return distribution and
$\phi(s)$, $s\in [0,1]$, is the so-called {\em risk aversion
  function\/}, a weighting function such that\footnote{Note that the
  treatment in \citep{Acerbi2002} is measure-based and therefore
  slightly different} 
\begin{enumerate}[(i)]
\item $\phi(p)\geq 0$,
\item $\int_0^1\phi(p)\, \dd p=1$,
\item $\phi'(p)\leq 0$. 
\end{enumerate}
Examples: VaR, ES\\
Replacing the last property with $\phi'(p)>0$ rules out risk-neutral
behaviour. \\
Spectral risk measures are coherent \citep{Acerbi2002}. 

\subsection{Representation of spectral risk measures}
\label{sec:repr-spectr-risk}

To prevent numerical instabilities involving the quantile function,
re-write spectral risk measures as follows:
\begin{itemize}
\item Integration by substitution: $\displaystyle \int_a^b
  g(\varphi(x)) \,\varphi'(x)\, \dd x = \int_{\varphi(a)}^{\varphi(b)}
  g(u)\, \dd u$.
\item Spectral risk measures: $\displaystyle -\int_0^1 \phi(p) \,
  F^{(-1)}(p)\, \dd p$
\item Set $\varphi(x)=F(x)$, $g(p) = \phi(p)\, F^{(-1)}(p)$.
\item Then:
  \begin{equation*}
    -\int_0^1 \phi(p)\, F^{(-1)}(p)\, \dd p = -\int_{-\infty}^\infty
    \phi(F(x))\, x\, f(x)\, \dd x.
  \end{equation*}
\end{itemize}


\subsection{Exponential spectral risk measures}
\label{sec:expon-risk-meas}

\begin{itemize}
\item Choose exponential utiliy function:
  $\displaystyle U(x) = -\e^{-k x}$, where $k>0$ is the Arrow-Pratt
  coefficient of absolute risk aversion
  (ARA).
\item Coefficient of absolute risk aversion: $\displaystyle R_A(x) =
  -\frac{U''(x)}{U'(x)} = k$
\item Coefficient of relative risk aversion: $\displaystyle R_R(x) = -
  \frac{x U''(x)}{U'(x)} = xk$
\item Weighting function $\phi(p) = \lambda \e^{-k(1-p)}$, where
  $\lambda$ is an unknown positive constant.
\item Set $\displaystyle\lambda = \frac{k}{1-\e^{-k}}$ to satisfy
  normalisation.
\item Exponential spectral risk measure:
  \begin{equation*}
    \rho_{\phi} = \int_0^1 \phi(p)\, F^{(-1)}(p)\, \dd p =
    \frac{k}{1-\e^{-k}} \int_0^1 \e^{-k(1-p)}\, F^{(-1)}(p)\, \dd p. 
  \end{equation*}
(If calculation of quantiles is a problem use change of variables
above.)
\item What exactly is the link between risk measure and utility?
    I think there is no direct link: the exponential risk measure is
   {\em inspired\/} by ARA utility.
\end{itemize}


%! Author = francis
%! Date = 30.10.20

\section{$D_1$ Operator}

The $D_1$ operator is given as
\begin{equation*}
    D_1 C_{X,Y}(F_X(x), F_Y(y)) = \p(Y\leq y|X=x).
\end{equation*}
In the context of the above notation, we obtain
\begin{align*}
  D_1 C_{R^s, R^F}(w, g(w)) &= \p(R_F\leq F_{F}^{(-1)}(g(w))|
  R_s=F_S^{(-1)}(w)) %
  = \p(V\leq g(w)| U=w)\\ %
  &= \frac{\p(U\in \dd w, V\leq g(w))} {\p(U\in \dd w)} %
  = \p(U\in \dd w| V\leq g(w)) %
  = \int_0^{g(w)} c(w,v)\, \dd v. 
\end{align*}
The last line can also be written as
\begin{equation*}
  \frac{\partial }{\partial w} C(w, g(w')) \big|_{w'=w}. 
\end{equation*}



We give an explicit equation of the $D_1$ operator for Archimedean copulae.

The $D_1$ operator is defined as the partial derivatives of the first input to the copula function,
so we fix the second argument while taking derivative with respect to the first, and then evaluate the function.
we have

\begin{align}
\left.\frac{\partial C\{v, g(w)\}}{\partial v} \right\vert_{v=w}
&=
\left.\frac{\partial  \phi^{-1}[\phi(v)+\phi\{g(w)\}]}
{\partial  [\phi(v)+\phi\{g(w)\}]}
\frac{\partial  [\phi(v)+\phi\{g(w)\}]}
{\partial  v}
\right\vert_{v=w}\\
&=
\left.\frac{\partial   \phi^{-1}[\phi(v)+\phi\{g(w)\}]}
{\partial [\phi(v)+\phi\{g(w)\}]}
\frac{\partial  \phi(v)}{\partial  v}
\right\vert_{v=w}\\
&=
\frac{\partial \phi^{-1}[\phi(w)+\phi\{g(w)\}]}
{\partial [\phi(w)+\phi\{g(w)\}]}
\frac{\partial  \phi(w)}{\partial  w}\\
&,
\text{where } g(w) = F_{R^F}\left\{\frac{F^{-1}_{R^S}(w)-r^h}{h}\right\}\\
\end{align}

\begin{table}
    \center
    \begin{tabular}{c c c c c}
        Function & Gumbel & Frank & Clayton & Independence\\
        $\phi(t)$    &
        $\{-\log(t)\}^\theta$ &
        $-\ln \left\{
        \frac{\exp(-\theta t)-1}
        {\exp(-\theta)-1}
        \right\}$&
        $\frac{1}{\theta}
        (t^{-\theta}-1)$
        & Same to Gumbel where $\theta=1$\\
        $\phi^{-1}(t)$ &
        $\exp(-t^{1/\theta})$ &
        $\frac{-1}{\theta}
        \log[1+ \exp(-t)\{\exp(-\theta)-1\}]$ &
        $(1+\theta t)^{-\frac{1}{\theta}}$
        & \\

        $\partial \phi(t)/\partial t$ &
        $\theta \frac{\phi(t)}{t\log(t)}$ &
        $\frac{\theta \exp(-\theta t)} {\exp(-\theta t)-1}$ &
        $-t^{-(\theta + 1)}$&
        \\
        $\partial \phi^{-1}(t)/\partial t$ &
        $\frac{-1}{\theta}t^{\frac{1}{\theta}-1}\phi^{-1}(t)$&
        $\frac{1}{\theta}\frac{\exp(-t)\{\exp(-\theta)-1\}}{1+\exp(-t)\{\exp(-\theta)-1\}}$&
        $\theta (1+\theta t)^{-\frac{1}{\theta}-1}$&
        \end{tabular}\label{tab:archcopula}
\end{table}








\section{Dependence}
\label{sec:dependence}

Dependence through copula (e.g.\ Student t, Clayton or Gumbel)

\subsection{Archimedean copulas}
\label{sec:archimedean-copulas}

\begin{itemize}
\item A well-studied one-parameter family of copulas are the {\bf 
    Archimedean copulas}. 
\item Let $\phi:[0,1]\rightarrow[0,\infty]$ be a
  continuous and strictly decreasing function with $\phi(1)=0$ and
  $\phi(0)\leq\infty$.
\item  We define the {\bf pseudo-inverse} of $\phi$ as 
  \begin{equation*}
    \phi^{(-1)}(t)=
    \begin{cases}
      \phi^{-1}(t), &0\leq t\leq \phi(0),\\
      0, &\phi(0)<t\leq\infty.
    \end{cases}
  \end{equation*}
\item If, in addition, $\phi$ is convex, then the following function
  is a copula: 
  \begin{equation*}
    C(u,v)=\phi^{(-1)}(\phi(u)+\phi(v)).
  \end{equation*}
  \vspace*{-\baselineskip}
\item Such copulas are called {\bf Archimedean copulas}, and the
  function $\phi$ is called an {\bf Archimedean copula generator}. 
\item Examples of Archimedean copulas are the {\bf Gumbel} and the
  {\bf Clayton} copulas:
  \begin{align*}
    C_{\theta,{\rm Gu}}(u,v) &= \exp\left\{-((-\ln u)^\theta + (-\ln
                               v)^\theta)^{1/\theta}\right\},& 1\leq \theta<\infty,\\
    C_{\theta,{\rm Cl}}(u,v)&= (u^{-\theta} + v^{-\theta}
                              -1)^{-1/\theta}, & 0<\theta<\infty. 
  \end{align*}
\item In the case of the Gumbel copula, the independence copula is 
  attained when $\theta=1$ and the comonotonicity copula is attained
  as $\theta\rightarrow\infty$. 
\item Thus, the Gumbel copula interpolates between independence and
  perfect dependence.  
\item In the case of the Clayton copula, the independence copula is
  attained as $\theta\rightarrow 0$, whereas the comonotonicity
  copula is attained as $\theta\rightarrow\infty$. 
\end{itemize}




\bibliographystyle{abbrvnamed} %
\bibliography{finance} %
\end{document}

%%% Local Variables: 
%%% mode: latex
%%% TeX-master: t
%%% End: 
