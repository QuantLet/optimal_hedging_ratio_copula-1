%! Author = francis
%! Date = 13.08.20

% Preamble
\documentclass[11pt, leqno]{article}
\usepackage[utf8]{inputenc}
\usepackage[english]{babel}
% Packages
\usepackage{amsfonts}
%\usepackage{amssymb}
\usepackage{amsmath}
\usepackage{amsthm}
\usepackage{hyperref}
\newtheorem{prop}{Proposition}[section]
\newtheorem{lemma}{Lemma}[section]
\newtheorem{definition}{Definition}[section]
\newtheorem{remark}{Remark}[section]
\newtheorem{prf}{Proof}[section]
\numberwithin{equation}{section}

% Document
\begin{document}
    \section{Another way to get $F_{X+Y}$ without using the $D_1$ Operator}\label{sec:D1alt}
The motivation of this section is that the current $D_1$ operator cannot give us a sensible result of $F_{X+Y}$.
This section provide an alternative route to involve the copula in the equation of $F_{X+Y}$.\\ \newline
    In principle, this alternative way should give us the same expression as by $D_1$ operator. \\ \newline
To keep things simple, we start with $F_Z$ where $Z = X+Y$.
    Assume the marginals $F_{X}$ and $F_{Y}$ are twice differentiable with respect to their input,
    and there exist a twice differentiable copula with respect to its two inputs $C_{XY} = F_{XY}$, we can write the p.d.f. of $Z$
    \begin{align}
    f_{Z}(z) = \int^\infty_{-\infty} f_{XY}(x, z-x)dx.
    \end{align} (Source!)

    We know by definition that the copula density $c_{XY}(u_1, u_2) = \frac{f_{XY}(F_X^{-1}(u_1), F_Y^{-1}(u_2))}{f_X(F_X^{-1}(u_1)) f_Y(F_Y^{-1}(u_2))}$,\\
    and $c_{XY}(u_1, u_2) = \frac{\partial^2 C_{XY}(u_1, u_2)}{\partial u_1 \partial u_2}$, we immediately have:
    \begin{align}
    f_{Z}(z) = \int^\infty_{-\infty} \label{eq:main}
    \frac{\partial^2 C_{XY}[F_X(x), F_Y(z-x)]}
         {\partial F_X(x) \partial F_Y(z-x)}f_X(x)f_Y(z-x)dx
    \end{align}.

    Let $u_1:=F_X(x)$ and $u_2:=F_Y(z-x)$, so $u_2=F_Y(z-F_X^{-1}(u_1))$, and
    \begin{enumerate}
        \item $\frac{\partial u_1}{\partial x}= f_X(x)$,
        \item $\frac{\partial u_2}{\partial x}= -f_Y(z-x)$,
        \item $\frac{\partial u_2}{\partial u_1} = -\frac{f_Y(z-x)}{f_X(x)}$, and
        \item $\frac{\partial^2 u_2}{\partial^2 u_1} =
                \frac{\partial f_Y(z-F^{-1}_X(u_1))}{\partial z - F_X^{-1}(u_1)}
                \frac{1}{f_X^2(F^{-1}(u_1))} +
                \frac{1}{f_X^3(F^{-1}(u_1))}\frac{\partial f_X(F^{-1}_X(u_1))}{\partial F_X^{-1}(u_1)}
                f_Y(z-F_X^{-1}(u_1))$
    \end{enumerate}

    Now we rewrite~\ref{eq:main}.
    \begin{align}
            f_{Z}(z) &= \int^\infty_{-\infty}
    \frac{\partial }{\partial u_2}
        \left(   \frac{\partial C_{XY}(u_1, u_2)}
         {\partial u_1}   \right)
            f_X(x)f_Y(z-x)dx\\
              &= \int^\infty_{-\infty}
    \frac{\partial }{\partial u_2}
        \left(   \frac{\partial C_{XY}(u_1, u_2)}
         {\partial u_1}   \right)\frac{\partial u_1}{\partial u_1}
            f_X(x)f_Y(z-x)dx\\
        &= \int^\infty_{-\infty} \frac{\partial^2 C_{XY}(u_1, u_2)}{\partial u_1^2} f^2_X(x)dx
    \end{align}
    For Archimedean copula $C_{XY}(u,v) = \phi^{-1}[\phi(u)+\phi(v)]$, we can further rewrite~\ref{eq:main}
        \begin{align}
            f_{Z}(z) &=
            \int^1_0 \bigg [
            \frac{\partial}{\partial u_1}
            \frac{\partial \phi^{-1}[\phi(u_1)+\phi(u_2)]}{\partial \phi(u_1)+\phi(u_2)}
            \left(
            \frac{\partial \phi(u_2)}{\partial u_1}-
            \frac{\partial \phi(u_1)}{\partial u_1}
            \right)
            \nonumber \\
            \qquad & +\frac{\partial \phi^{-1}[\phi(u_1)+\phi(u_2)]}{\partial \phi(u_1)+\phi(u_2)}
            \left(
            \frac{\partial^2 \phi(u_2)}{\partial u_1^2}-
            \frac{\partial^2 \phi(u_1)}{\partial u_1^2}
            \right)
            \bigg ]f^2_X(F^{-1}_X(u_1))du_1 \label{eq:fZ}
    \end{align}
    Let's observe the equation.
    \begin{enumerate}
    \item $\frac{\partial}{\partial u_1}
            \frac{\partial \phi^{-1}[\phi(u_1)+\phi(u_2)]}{\partial \phi(u_1)+\phi(u_2)}$
    \item $\frac{\partial \phi(u_2)}{\partial u_1} = \frac{\partial \phi(u_2)}{\partial u_2}\frac{\partial u_2}{\partial u_1}$
    \item $\frac{\partial^2 \phi(u_2)}{\partial u_1^2} = \frac{\partial^2 \phi(u_2)}{\partial u_1 \partial u_2}+
    \frac{\partial^2 u_2}{\partial u_1^2}\frac{\partial \phi(u_2)}{\partial u_2}$
    \end{enumerate}
    The above parts are solvable case by case depending on which copula is chosen.
    The equation~\ref{eq:fZ} is now ready to be numerically solved by plugging in $x=F^{-1}(u_1)$ and $u_2 = F_Y(z-F^{-1}_X(u_1))$ to it.
\end{document}