
\subsection{Copulae}\label{subsec:copulae}
As we saw from the last section, risk measures we considered are all functionals of joint distribution of $r^S$ and $r^F$.
We test different copulae for their ability to model joint distribution of crypto-related assets returns.
We consider Gaussian-, $t$-, Frank-, Gumbel-, Clayton-, Plackett-, mixture, and factor copula.
This hedging exercise concerns only portfolios with two assets, we only present the bivariate version of copulae and some important features of a copula,
they include Kendall's $\tau$, Spearman's $\rho$,
upper tail dependence $\lambda_U \overset{\mathrm{def}}{=}  \lim_{q \rightarrow 1^-} \p\{X > F_{X}^{(-1)}(q)|Y > F_{Y}^{(-1)}(q)\}$ and
lower tail dependence $\lambda_L \overset{\mathrm{def}}{=}  \lim_{q \rightarrow 0^+} \p\{X \leq F_{X}^{(-1)}(q)|Y \leq F_{Y}^{(-1)}(q)\}$.
Furthermore, we denote the Fr{\'e}chet-Hoeffding lower bound as $\bm{W}$, product copula as $\bm{\Pi}$, and the Fr{\'e}chet-Hoeffding upper bound as $\bm{M}$,
they represent cases of perfect negative dependence, independence, and perfect positive dependence respectively.
For further detail, we refer readers to \citet{joe1997multivariate} and \citet{Nelsen1999}.
See also \citet{hardle2010copulis}.

\subsubsection{Elliptical Copulae}\label{sec:ellpitical-copulae}
Elliptical copulae are dependence structure associated with elliptical distributions.
The bivariate Gaussian copula is:
    \begin{align}
        \bm{C}(u,v) &= \Phi_{2, \rho}\{\Phi^{-1}(u), \Phi^{-1}(v)\} \nonumber \\
                    &= \int_{-\infty}^{\Phi^{-1}(u)}
                       \int_{-\infty}^{\Phi^{-1}(v)}
                       \frac{1}{2\pi\sqrt{1-\rho^2}}
                       \exp{\left(
                       \frac{s^2-2\rho st+t^2}{2(1-\rho^2)}
                       \right)} ds dt
        \end{align}
where $\Phi_{2, \rho}$ is the cdf of bivariate Normal distribution with zero mean, unit variance, and correlation $\rho$ \francis{\em the rho we use in risk measure is $\rho_{\text{something}}$. This should remove ambiguity.},
and $\Phi^{-1}$ is quantile function univariate standard normal distribution.
The Gaussian copula density is
\begin{align}
    \bm{c}_\rho(u,v) &= \frac{\bm{\varphi}_{2,\rho}\{\Phi^{-1}(u), \Phi^{-1}(v)\}}
                        {\varphi\{\Phi^{-1}(u)\} \cdot \varphi\{\Phi^{-1}(v)\}} \nonumber \\
                &= \frac{1}{2\pi\sqrt{1-\rho^2}}\exp\left(
                   -\frac{u^2 - 2\rho uv + v^2}{2(1-\rho^2)}
                   \right),
    \end{align}
where $\bm{\varphi}_{2,\rho}(\cdot)$ is the pdf of $\Phi_{2, \rho}$,
and $\varphi(\cdot)$ the standard normal distribution pdf.\medskip

The Kendall's $\tau_K$ and Spearman's $\rho_S$ of a bivariate Gaussian Copula are
    \begin{align}
        \tau_K(\rho) = \frac{2}{\pi}\arcsin\rho
        \end{align}
    \begin{align}
        \rho_S(\rho) = \frac{6}{\pi}\arcsin\frac{\rho}{2}
        \end{align}\medskip

The $t$-Copula has a form
\begin{align}
        \bm{C}(u,v) &= \bm{T}_{2, \rho, \nu}\{T^{-1}_\nu(u), T^{-1}_\nu(v)\} \nonumber \\
            &= \int_{-\infty}^{T^{-1}_\nu(u)}
               \int_{-\infty}^{T^{-1}_\nu(v)}
            \frac{\Gamma\left(\frac{\nu+2}{2}\right)}
            {\Gamma\left(\frac{\nu}{2}\right)\pi\nu\sqrt{1-\rho^2}}\\
           & \left(
        1+\frac{s^2-2st\rho+t^2}{\nu}
        \right)^{-\frac{\nu+2}{2}} ds dt,
    \end{align}
where $\bm{T}_{2, \rho, \nu}(\cdot, \cdot)$ denotes the cdf of bivariate $t$ distribution with scale parameter $\rho$ and degree of freedom $\nu$,
$T^{-1}_\nu(\cdot)$ is the quantile function of a standard $t$ distribution with degree of freedom $\rho$.

The copula density is
\begin{align}
    \bm{c}(u,v) &= \frac{\bm{t}_{2, \rho, \nu}\{T^{-1}_\nu(u), T^{-1}_\nu(v)\}}
    {t_\nu\{T^{-1}_\nu(u)\}\cdot t_\nu\{T^{-1}_\nu(v)\}},
    \end{align}
where $\bm{t}_{2,\rho, \nu}$ is the pdf of $\bm{T}_{2, \rho, \nu}(\cdot, \cdot)$,
and $t_\nu$ the density of standard $t$ distribution.\medskip

Like all the other elliptical copulae, $t$ copula's Kendall's $\tau$ is identical to that of Gaussian copula (Demarta and reference therein).

\subsubsection{Archimedean Copulae}\label{sec:archimedean-copula}
The Archimedean copulae forms a large class of copulae with many convenient features.
In general, they take a form
\begin{align}
    \bm{C}(u,v)= \psi^{-1}\{\psi(u), \psi(v)\},
    \end{align}
where $\psi:[0,1] \rightarrow [0,\infty)$ is a continuous, strictly decreasing and convex function such that
$\psi(1)=0$ for any permissible dependence parameter $\theta$. $\psi$ is also called generator.
$\psi^{-1}$ is the inverse the generator.\medskip

The Frank copula (B3 in \citet{joe1997multivariate}) is a radial symmetric copula and cannot produce any tail dependence.
It takes the form
\begin{align}
    \bm{C}_{\theta}(u,v) &= \frac{1}{\theta}
    \log \left\{
    1 + \frac{(e^{-\theta u}-1)(e^{-\theta v}-1)}{e^{-\theta}-1}
    \right\}
    \end{align}
where $\theta \in [0, \infty]$ is the dependency parameter.
$\bm{C}_{-\infty} = \bm{M}$, $\bm{C}_1 = \bm{\Pi}$, and $\bm{C}_\infty = \bm{W}$.

The Copula density is
\begin{align}
    \bm{c}_{\theta}(u,v) &= \frac{\theta e^{\theta(u+v)(e^\theta-1)}}
    {\left\{e^\theta-e^{\theta u}-e^{\theta v}+e^{\theta (u+v)}\right\}^2}
    \end{align}\medskip

Frank copula has Kendall's $\tau$ and Spearman's $\rho$ as follow:
\begin{align}
    \tau_K(\theta) = 1-4\frac{D_1\{-\log(\theta)\}}{\log(\theta)},
    \end{align}
and
\begin{align}
    \rho_S(\theta) = 1-12\frac{D_2\{-\log(\theta)\} - D_1\{\log(\theta)\}}{\log(\theta)},
    \end{align}
where $D_1$ and $D_2$ are the Debye function of order 1 and 2.
Debye function is $D_n = \frac{n}{x^n}\int_0^x\frac{t^n}{e^t-1}dt$.\medskip

Gumbel copula (B6 in \citet{joe1997multivariate}) has upper tail dependence with the dependence parameter
$\lambda^U = 2-2^{\frac{1}{\theta}}$ and displays no lower tail dependence.
\begin{align}
    \bm{C}_{\theta}(u,v) &= \exp{-\{
    (-\log(u))^\theta +(-\log(v))^\theta
    \}^{\frac{1}{\theta}}},
    \end{align}
where $\theta \in [1,\infty)$ is the dependence parameter.\medskip
While Gumbel copula cannot model perfect counter dependence (ref), $\bm{C}_{1} = \bm{\Pi}$ models the independence,
and $\lim\limits_{\theta \to \infty} \bm{C}_\theta = \bm{W}$ models the perfect dependence.

%The copula density takes the form
%\begin{align}
%        f
%    \end{align}

  \begin{align}
    \tau_K(\theta) =\frac{\theta-1}{\theta}
    \end{align}

The Clayton copula, by contrast to Gumbel copula,
generates lower tail dependence in a form $\lambda^L = 2^{-\frac{1}{\theta}}$,
but cannot generate upper tail dependence.\medskip

The Clayton copula takes the form
\begin{align}
    \bm{C}_{\theta}(u,v) &= \left\{
    \max(u^{-\theta}+v^{-\theta}-1,0)\right\}^{-\frac{1}{\theta}},
    \end{align}
where $\theta \in (-\infty, \infty)$ is the dependency parameter.
$\lim\limits_{\theta \to -\infty} \bm{C}_\theta = \bm{M}$, $\bm{C}_0 = \bm{\Pi}$, and $\lim\limits_{\theta \to \infty} \bm{C}_\theta = \bm{W}$.\medskip

Kendall's $\tau$ to this copula dependency is 
\begin{align}
    \tau_K(\theta) =\frac{\theta}{\theta+2}.
    \end{align}\medskip

\subsubsection{Mixture Copula}\label{sec:mixture-copula}
Mixture copula is a linear combination of copulae. 
For a 2-dimensional random variable $\bm{X}=(X_1,X_2)^\top$,
its distribution can be written as linear combination $K$ copulae
\begin{align}
    \p(X_1 \leq x_1, X_2 \leq x_2) = \sum_{k=1}^K p^k \cdot \bm{C}^{(k)}\{F^{(k)}_{X_1}(x_1;\bm{\gamma}^{(k)}_1),
    F^{(k)}_{X_2}(x_2;\bm{\gamma}^{(k)}_2); \bm{\theta^{(k)}}\}
    \end{align}
where $p^{(k)} \in [0,1]$ is the weight of each component,
$\bm{\gamma}^{(k)}$ is the parameter of the marginal distribution in the $k^\text{th}$ component,
and $\bm{\theta^{(k)}}$ is the dependence parameter with the copula of the $k^\text{th}$ component.
The weights add up to one $\sum_{k=1}^K p^{(k)}=1$. \medskip

We deploy a simplified version of the above representation by assuming the margins of $\bm{X}$ are not mixture.
By Sklar's theorem one may write
\begin{align}
    \bm{C}(u,v)= \sum_{k=1}^K p^{(k)} \cdot \bm{C}^{(k)}\{F^{-1}_{X_1}(u),
    F^{-1}_{X_2}(v); \bm{\theta^{(k)}}\}.
    \end{align}\medskip

The copula density is again a linear combination of copula density
\begin{align}
    \bm{c}(u,v)= \sum_{k=1}^K p^{(k)} \cdot \bm{c}^{(k)}\{F^{-1}_{X_1}(u),
    F^{-1}_{X_2}(v); \bm{\theta^{(k)}}\}.
    \end{align}\medskip

While Kendall's $\tau$ of mixture copula is not known in close form,
the Spearman's $\rho$ is

\begin{proposition}
    Let $\rho_S^{(k)}$ be the Spearman's $\rho$ of the $k^\text{th}$ component and $\sum_{k=1}^K p^{(k)}=1$ holds,
    the Spearman's $\rho$ of a mixture copula is
    \begin{align}
        \rho_S = \sum_{k=1}^K p^{(k)} \cdot \rho_S^{(k)}
        \end{align}
    \end{proposition}

\begin{proof}
    Spearman's $\rho$ is defined as \citep{Nelsen1999}
    \begin{align}
        \rho_S = 12 \int_{\mathbb{I}^2} \bm{C}(s,t) ds dt - 3.
        \end{align}
    Rewrite the mixture copula into sumation of components
       \begin{align}
        \rho_S = 12 \int_{\mathbb{I}^2} \sum_{k=1}^K p^{(k)} \cdot \bm{C}^{(k)}(s,t) ds dt - 3.
        \end{align}
    \end{proof}

\begin{example}
    The Fr{\'e}chet class can be seen as an example of mixture copula.
    It is a convex combinations of $\bm{W}$, $\bm{\Pi}$, and $\bm{M}$ \citep{Nelsen1999}
    \begin{align}
        \bm{C}_{\alpha, \beta}(u,v)
        = \alpha \bm{M}(u,v) +
        (1-\alpha-\beta)\bm{\Pi}(u,v)
        +\beta \bm{W}(u,v),
        \end{align}
    where $\alpha$ and $\beta$ are the dependence parameters, with $\alpha, \beta \geq 0$ and
    $\alpha+\beta \leq 1$.
    Its Kendall's $\tau$ and Spearman's $\rho$ are
    \begin{align}
        \tau_K(\alpha, \beta) = \frac{(\alpha - \beta)(\alpha+\beta+2)}{3}
        \end{align}
    , and
    \begin{align}
        \rho_S(\alpha, \beta) = \alpha - \beta
        \end{align}
    \end{example}\medskip
%Example 2 Gumbel-Clayton mixture
%Example 3 Hu 2006.

We use a mixture of Gaussian and independent copula in our analysis.
We write the copula
\begin{align}
    \bm{C}(u,v) = p\cdot \bm{C}^\text{Gaussian}(u,v) + (1-p)(uv).
    \end{align}
The corresponding copula density is
\begin{align}
    \bm{c}(u,v) = p\cdot \bm{c}^\text{Gaussian}(u,v) + (1-p).
    \end{align}

This mixture allows us to model how much "random noise" appear in the dependency structure.
In this hedging exercise, the structure of the "random noise" is not of our concern nor we can
hedge the noise by a two-asset portfolio.
However, the proportion of the "random noise" does affect the distribution of $r^h$,
so as the optimal hedging ratio $h^\ast$.
One can consider the mixture copula as a handful tool for stress testing.
Similar to this Gaussian mix Independent copula,
t copula is also a two parameter copula allow us to model the noise,
but its interpretation of parameters is not as intuitive as that of a mixture.
The mixing variable $p$ is the proportion of a manageable (hedgable) Gaussian copula,
while the remaining proportion $1-p$ cannot be managed.

\subsection{Other Copula}\label{subsec:other-copula}
The Plackett copula has an expression
\begin{align}
    \bm{C}_{\theta}(u,v) &= \frac{1+(\theta-1)(u+v)}{2(\theta-1)}
                         - \frac{\sqrt{\{
    1+(\theta-1)(u+v)\}^2 - 4uv\theta(\theta-1)}}{2(\theta-1)}
    \end{align}
\begin{align}
    \rho_S(\theta) = \frac{\theta+1}{\theta-1} - \frac{2\theta \log \theta}{(\theta-1)^2}
    \end{align}\medskip

We include Placket copula in our analysis as it possesses a special property,
the cross-product ratio is equal to the dependence parameter
\begin{align}
    &\phantom{=} \frac{\p(U \leq u, V \leq v) \cdot \p(U > u, V > v)}
    {\p(U \leq u, V > v) \cdot \p(U > u, V \leq v)}\nonumber\\
    &= \frac{\bm{C}_\theta(u,v)\{1-u-v+\bm{C}_\theta(u,v)\}}{\{u-\bm{C}_\theta(u,v)\}\{v-\bm{C}_\theta(u,v)}\nonumber\\
    &= \theta.
    \end{align}\medskip

That is, the dependence parameter is equal to the ratio between number of concordence pairs and number of discordence pairs of a bivariate random variable.
