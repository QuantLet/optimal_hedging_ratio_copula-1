\section{Optimal hedge ratio $h^*$}
\label{sec:optimal-hedge-ratio}

\natp{\em [Try to avoid mathematical notation in section
  headings. Also, avoid mathematical notation at the beginning of a
  sentence. Formulate {\bf carefully}! For example: ``The hedged
  portfolio is $Z=X-hY$.'' is not precise. First, $X$ and $Y$ have not
  been introduced. Second $Z$ refers to the portfolio return. Third,
  $Z$ refers to {\em any\/} portfolio. It is not explained why this
  would be a hedge. Below is a suggestion, now go through the text and
  formulate carefully what you want to say. This takes time...]}

\natp{Let $r_S$ and $r_F$ be the returns of the spot and futures price
  of an asset. The hedged position is a portfolio with return $r_h =
  r_S - h r_F$, where the amount $h$ of futures to short is chosen to
  minimise the risk of the exposure in the spot. (was: 
Consider the problem of an optimal
hedge ratios by risk measures and dependence
structures.
The hedged portfolio is $Z = X - h Y$.
It involves returns of two assets $X$ and $Y$.
This is the situation where we long one unit of asset $X$ and short $h$ unit of asset $Y$.
$h$ is called the hedge ratio.)}
We denote the risk measure of the returns of the hedged portfolio $Z$ as $\rho(Z)$. \medskip
\natp{\em [Again: Introduce $\rho$ as being a risk measure on financial
  positions, i.e., a mapping from a financial position to a real
  number (see e.g.\ McNeil, Chapter 6 or the book by F\"ollmer Schied).]} 

The optimal hedge ratio is $h^\ast = \argmin_h \rho(Z)$, that is the best ratio that can minimize the risk of a hedged portfolio measured in terms of $\rho$ . \medskip

\natp{\em [Introduce the definition of copulas here. Also, this is a
  good place to motivate the separate modelling of dependence and
  margins. First, explain that the ability to hedge is driven by the
  dependence in $r_S$ and $r_F$. And so on. ]}
The distribution function of $Z$ can be expressed in terms of the
copula and the marginal distributions as Proposition \ref{prop:dfrh}
result shows (this is a corrected version of Corollary 2.1 of
\citep{barbi2014copula}). For practical applications, it is numerically
faster and more stable to use additional information about the
specific copula and marginal distributions. \natp{\em [What do you
  mean by additional information? Fast than what?]} We therefore derive
semi-analytic formulae for a number of special cases, such as the
Gaussian-, Student $t$-, normal inverse Gaussian (NIG) and Archimedean
copulas in Section \ref{subsec:copulae}.

\natp{\em [What is the purpose of the parameters $w$ and $\lambda$
  below? Leave them out here and use them where they are needed.]}

\begin{proposition}
  \label{prop:dfrh}
  Let $X$ and $Y$ be two real-valued \natp{continuous} random
  variables on \natp{a (was: the same)}
  probability space $(\Omega, \mathcal A, \p)$ with \natp{(delete: corresponding)}
  absolutely continuous copula $C_{X, Y}$ and
  \natp{(delete: continuous)} marginal distribution functions $F_{X}$
  and $F_{Y}$. Then, the distribution function of of $Z$ is given by 
  \begin{equation}
    \label{eq:3}
    F_{Z}(z) = 1- \int^1_0 D_1 C_{X, Y}
    \left[ u, F_{Y} \left\{ \frac{F^{(-1)}_{X}(u)-z}{h} \right\}
    \right]\, d u.
  \end{equation}
\end{proposition}
Here, $F^{(-1)}$ denotes the inverse of $F$, i.e., the quantile
function. \natp{\em [Spell out inverse function, with $\inf$.]}

Here $D_1 C(u,v)=\displaystyle \frac{\partial}{\partial u} C(u,v)$,
which is easily shown to fulfil, see e.g.\ Equation (5.15) of
\citep{McNeil2005}:
\begin{equation}
  \label{eq:1}
  D_1 C_{X,Y}(F_X(x), F_Y(y)) = \p(Y\leq y|X=x).
\end{equation}
\begin{proof}
  \natp{\em [Use blackboard E as expectation operator, i.e., $\E$.]}
  Using the identity (\ref{eq:1}) gives
  \begin{align*}
    F_{Z}(z) &= \p(X - h Y\leq z) %
                 = \mbox{\sf E}\left\{\p\left(Y\geq \frac{X-z}{h}\Big|
                 X\right)\right\}\\[10pt]
               &= 1-\mbox{\sf E}\left\{\p\left(Y\leq \frac{X-z}{h}\Big|
                 X\right)\right\}% \\[10pt]
               = 1- \int_0^1 D_1 C_{X, Y}\left[u,
                 F_{Y}\left\{\frac{F^{(-1)}_{X}(u) -
                 z}{h}\right\}\right]\, d u.
  \end{align*}
  \end{proof}


In addition to~\cite{barbi2014copula} we propose a more handy
expression for the density of $Z$. \natp{\em [Is this used later? If
  not, thenn possibly remove. If it stays, then make this a
  corollary. Fix punctuation below. Also, please double-check the
  ``+'' signs in the second equation.]}

\begin{prop} Given the formulation of the above portfolio, the density of $Z$ can be written as
  \begin{align}
  f_{Z}(z) &= \left|\frac{1}{h}\right|\int_0^1 c_{X, Y} \left[
  F_{Y}\left\{\frac{F^{(-1)}_{X}(u)-z}{h}\right\}, u
  \right]
   \cdot
  f_{Y}
  \left\{\frac{F^{(-1)}_{X}(u)-z}{h}\right\} du \label{eq:density1}
  \end{align}, or
    \begin{align}
      f_{Z}(z)
      = \int_0^1 c_{X, Y} \left[
      F_{X}\left\{z + h F^{(-1)}_{Y}(u)\right\}, u
      \right]
       \cdot
      f_{X}
      \left\{
      z+ hF^{(-1)}_{Y}(u).
      \right\} du\label{eq:density2}
  \end{align}
  \end{prop}
The two expression are equivalent, one can use any of them to get the
density of $Z$. \natp{\em [Reformuate... why ``get the density''? They
  {\bf are} the density, no?]}
Notice that the density of $Z$ in the above proposition is readily accessible as long as we have
the copula density and the marginal densities.
The proof and a generic expression can be found in the
appendix. \medskip

\natp{\em [Portfolios. Not aware that portfolio is a Latin word.]}

In this work, we consider two portfolii: $R^h = R^{\text{BTC}} - h R^{\text{future}}$ and $R^h = R^{\text{BTC}} - h R^{\text{CRIX}}$.

\natp{\em [For now, stick to $r_S$ and $r_F$. This is still the
  general part, so no need to mention Bitcoin, CRIX, etc. here. Also,
  wasn't the idea to hedge CRIX with the future? I don't see why CRIX
  would be used to hedge Bitcoin when futures are readily
  available. Entering into futures contracts requires no notional,
  which makes them ideal for hedging.]}

%OLD
%\section{Methodology}\label{sec:methodology}
%Following \citet{barbi2014copula}, we consider the problem of optimal
%hedge ratios by extending the commonly known minimum variance hedge
%ratio to more general risk measures and dependence
%structures.\medskip
%
%Hedge portfolio: $R_t^h = R_t^S - h R_t^F$, involving returns of spot
%and future contract and where $h$ is the hedge ratio.\medskip
%
%The optimal hedge ratio is
%\begin{align}
%    h^\ast = \argmin_h \rho_\phi(s,h),
%    \end{align}
%for given
%confidence level $1-s$ (if applicable, e.g.\ in the case of VaR, ES),
%where $\rho_\phi$ is a spectral risk measure with weighting function
%$\phi$ (see below).
%In other words, our task is to search for the optimal $h$ which can minimize a particular risk measure.
%We call the risk measure being used in search process of $h^\ast$ risk reduction objective.
%This naming is to differentiate the risk objective and risk outcome.
%One can see from result section that the $h^\ast$ which minimize a particular risk measure in training does not
%necessarily minimize the risk measure in testing data.
%For example in table \ref{OOSRHVaR99}, the best performing risk reduction objective to reduce out-of-sample Value-at-Risk 99\% is
%exponential risk measure $k=10$. \medskip
%
%The distribution function of $Z$ can be expressed in terms of the
%copula and the marginal distributions as Proposition \ref{prop:dfrh}
%result shows (this is a corrected version of Corollary 2.1 of
%\citep{barbi2014copula}). For practical applications, it is numerically
%faster and more stable to use additional information about the
%specific copula and marginal distributions. We therefore derive
%semi-analytic formulas for a number of special cases, such as the
%Gaussian-, Student $t$-, normal inverse Gaussian (NIG) and Archimedean
%copulas in Section \ref{sec:dependence}.
%
%\begin{proposition}
%  \label{prop:dfrh}
%  Let $X$ and $Y$ be two real-valued random variables on the same
%  probability space $(\Omega, \mathcal A, p)$ with corresponding
%  absolutely continuous copula $C_{X, Y}(w,\lambda)$ and
%  continuous marginals $F_{X}$ and $F_{Y}$. Then, the distribution
%  of of $Z$ is given by
%  \begin{equation}
%    \label{eq:3}
%    F_{Z}(x) = 1- \int^1_0 D_1 C_{X, Y}
%    \left( u, F_{Y} \left( \frac{F^{(-1)}_{X}(u)-x}{h} \right)
%    \right)\, d u.
%  \end{equation}
%\end{proposition}\medskip
%Here $D_1 C(u,v)=\displaystyle \frac{\partial}{\partial u} C(u,v)$,
%which is easily shown to fulfil, see e.g.\ Equation (5.15) of
%\citep{McNeil2005}:\footnote{%
%  Let $F_X(x)=u$, $F_Y(y)=v$. Then, formally,
%  \begin{align*}
%    \frac{\partial}{\partial F_X(x)} C(F_X(x), F_Y(y)) %
%    &= \frac{\partial}{\partial F_X(x)} \p(U\leq F_X(x),
%      V\leq F_Y(y)) %
%      = \p(U\in d F_X(x), V\leq F_Y(y))\\ %
%    &= \p(V\leq F_Y(y)| U = F_X(x))\cdot \p(U \in d
%      F_X(x)) %
%      = \p(Y\leq y|X=x)\cdot \p(U\in d u)\\ %
%    &= \p(Y\leq y|X=x).
%  \end{align*}}
%\begin{equation}
%  \label{eq:1}
%  D_1 C_{X,Y}(F_X(x), F_Y(y)) = \p(Y\leq y|X=x).
%\end{equation}
%\begin{proof}
%  Using the identity (\ref{eq:1}) gives
%  \begin{align*}
%    F_{Z}(x) &= \p(R^s - h Y\leq x) %
%                 = \E\left[\p\left(Y\geq \frac{X-x}{h}\Big|
%                 X\right)\right]\\[10pt]
%               &= 1-\E\left[\p\left(Y\leq \frac{X-x}{h}\Big|
%                 X\right)\right]% \\[10pt]
%               = 1- \int_0^1 D_1 C_{X, Y}\left(u,
%                 F_{Y}\left(\frac{F^{(-1)}_{X}(u) -
%                 x}{h}\right)\right)\, d u.
%  \end{align*}
%\end{proof}\medskip
%
%In addition to \cite{barbi2014copula} we propose an expression for the density of $Z$
%
%\begin{proposition} With the same setting of the above proposition, the density of $Z$ can be written as
%  \begin{align}
%  f_{Z}(y) &= \left|\frac{1}{h}\right|\int_0^1 c_{X, Y} \left[u,
%  F_{Y}\left\{\frac{F^{(-1)}_{X}(u)-y}{h}\right\}
%  \right]
%   \cdot
%  f_{Y}
%  \left\{\frac{F^{(-1)}_{X}(u)-y}{h}\right\} du \label{eq:density1}
%  \end{align}, or
%    \begin{align}
%      f_{Z}(y)
%      = \int_0^1 c_{X, Y} \left[u,
%      F_{X}\left\{y + h F^{(-1)}_{Y}(u)\right\}
%      \right]
%       \cdot
%      f_{X}
%      \left\{
%      y+ hF^{(-1)}_{Y}(u)
%      \right\} du.\label{eq:density2}
%  \end{align}
%  \end{proposition}
