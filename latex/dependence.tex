
Dependence through copula (e.g.\ Student t, Clayton or Gumbel)

\subsection{Archimedean copulas}
\label{sec:archimedean-copulas}

\begin{itemize}
\item A well-studied one-parameter family of copulas are the {\bf 
    Archimedean copulas}. 
\item Let $\phi:[0,1]\rightarrow[0,\infty]$ be a
  continuous and strictly decreasing function with $\phi(1)=0$ and
  $\phi(0)\leq\infty$.
\item  We define the {\bf pseudo-inverse} of $\phi$ as 
  \begin{equation*}
    \phi^{(-1)}(t)=
    \begin{cases}
      \phi^{-1}(t), &0\leq t\leq \phi(0),\\
      0, &\phi(0)<t\leq\infty.
    \end{cases}
  \end{equation*}
\item If, in addition, $\phi$ is convex, then the following function
  is a copula: 
  \begin{equation*}
    C(u,v)=\phi^{(-1)}(\phi(u)+\phi(v)).
  \end{equation*}
  \vspace*{-\baselineskip}
\item Such copulas are called {\bf Archimedean copulas}, and the
  function $\phi$ is called an {\bf Archimedean copula generator}. 
\item Examples of Archimedean copulas are the {\bf Gumbel} and the
  {\bf Clayton} copulas:
  \begin{align*}
    C_{\theta,{\rm Gu}}(u,v) &= \exp\left\{-((-\ln u)^\theta + (-\ln
                               v)^\theta)^{1/\theta}\right\},& 1\leq \theta<\infty,\\
    C_{\theta,{\rm Cl}}(u,v)&= (u^{-\theta} + v^{-\theta}
                              -1)^{-1/\theta}, & 0<\theta<\infty. 
  \end{align*}
\item In the case of the Gumbel copula, the independence copula is 
  attained when $\theta=1$ and the comonotonicity copula is attained
  as $\theta\rightarrow\infty$. 
\item Thus, the Gumbel copula interpolates between independence and
  perfect dependence.  
\item In the case of the Clayton copula, the independence copula is
  attained as $\theta\rightarrow 0$, whereas the comonotonicity
  copula is attained as $\theta\rightarrow\infty$. 
\end{itemize}


\subsection{Elliptical Copulae}\label{subsec:elliptical-copulae}

\begin{definition}
    Elliptical Distribution.
    The $d$-dimensional random vector $\pmb{y}$ has an elliptical distribution if and only if the characteristic function
    $\pmb{t} \mapsto \mathbb{E}\{\exp(i\pmb{t}^\top \pmb{y})\}$ with $\pmb{t} \in \mathbb{R}^d$ has the representation
    \begin{align}
        \phi_g(\pmb{t}; \pmb{\mu}, \pmb{\Sigma}, \pmb{\nu}) = \exp(i\pmb{t}^\top\pmb{\mu})g(\pmb{t}^\top\pmb{\Sigma}\pmb{t};\pmb{\nu})
        \end{align}
    where $g(\cdot;\nu):[0, \infty[ \mapsto \mathbb{R}$, $\nu \in \mathbb{R}^d$, and $\Sigma$ is a symmetric positive semidefinite $d\times d$-matrix.
    \end{definition}

The function $g(\cdot; \nu)$ is known as characteristic generator, whereas $\pmb{\nu}$ is parameter that determines the shape, in particular the tai index of the distribtion.

\begin{corollary} \citep[equation 2.12]{fang2018symmetric}
    If $\pmb{y}$ follows an elliptical distribution, then $\pmb{y}$ has a stochastic representation
    \begin{align}\label{eq:stochastic-representation}
        \pmb{y} = \pmb{\mu} + r\pmb{A}^\top \pmb{u},
        \end{align}
    where $r \in \mathbb{R}_+$ is independent of
    $\pmb{u}$
%\footnote{$\pmb{u}$ is uniformly distributed on $S_d = \{\pmb{u} \in \mathbb{R}^d s.t. ||\pmb{u}|| = 1\}$},
    , and $\pmb{A}^\top\pmb{A}=\pmb{\Sigma}$.
    \end{corollary}

\begin{table}[ht]
    \center
    \begin{tabular}{lll}
    Distribution & $r \sim$ & $g(\pmb{t})$\\ \hline
    Gaussian & $\chi_n$ &
        \end{tabular}
    \caption{Generators of Elliptical Distributions summarised from~\cite[Chapter 2]{fang2018symmetric}}
    \label{tab:table}
\end{table}


\subsection{Gaussian Copula}\label{subsec:Gaussian-copula}
The Gaussian or Normal copula is
\begin{align}
    C^{Ga}_\Sigma(x) = \frac{1}{(2\pi)^{i/2} |\Sigma|^{1/2}}
    \int_{-\infty}^{\Phi^{-1}(x_1)} \dots \int_{-\infty}^{\Phi^{-1}(x_n)}
    \exp \left\{
    -\frac{1}{2}y^\top \Sigma^{-1}y
    \right\}
    dy_1 \dots dy_n
    \end{align}

\subsection{t-copulae}\label{subsec:t-copulae}
The t copula is to represent the dependency structure by t distribution~\citep{fang2002meta, embrechts2002correlation}.
\cite{demarta2005t} extend this thinking to skewed t copula and rouped t copula to allow more flexibility in the modelling of dependency structure.
T-copula is based on the Gaussian mixture representation (see equation~\ref{eq:stochastic-representation}) of a multivariate t distribution,
and based on that skewed t copula and grouped t copula are constructed.

\subsubsection{Vanilla t-copula}\label{subsec:vanilla-t-copula}
The t-copula is
\begin{align}
    C^t_{\nu, \Sigma}(x) = \frac{1}{(2\pi)^{i/2} |\Sigma|^{1/2}}
    \int_{-\infty}^{t_\nu^{-1}(x_1)} \dots \int_{-\infty}^{t_\nu^{-1}(x_n)}
    \frac{\Gamma\left\{ \frac{\nu + i}{2}\right\}}{\Gamma \left\{\frac{\nu}{2}\right\} (\pi \nu)^{i/2}|\Sigma|^{1/2}}
    \left(
    1+ \frac{y^\top \Sigma^{-1}y}{\nu}
    \right)^{-\frac{\nu + i}{2}}
    dy_1 \dots dy_n
    \end{align}

\subsubsection{Skewed t copula}\label{subsec:skewed-t-copula}
Mean variance mixture

\subsubsection{Double t copula}
under construction

\subsubsection{Normal Inverse Gaussian Copual}
under construction






%\subsection{Extreme-value copulae}\label{subsec:extreme-value-copulae}
