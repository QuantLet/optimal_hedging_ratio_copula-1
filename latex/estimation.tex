%! Author = francis
%! Date = 30.10.20


\subsection{Two-Stage Estimation}\label{subsec:two-stage-estimation}
~\cite{joe2005asymptotic} study the efficiency of a two-stage estimation procedure of copula estimation.
The authors also call this method inference function for margins IFM.

\textbf{Pros}
\begin{enumerate}
    \item Almost as efficient as MLE methods but easier to be implemented
    \item Yields an asymptotically Gaussian, unbiased estimate
\end{enumerate}

\textbf{Cons}
\begin{enumerate}
    \item Subject to specification of marginals \cite{kim2007comparison}
\end{enumerate}

Our data
\begin{align}
    \pmb{y} = \begin{bmatrix}
                  y_{11} & \cdots & y_{1i}\\
                  \vdots & \ddots & \vdots \\
                  y_{n1} & \cdots & y_{ni}
                  \end{bmatrix}
    \end{align}
Let $F$ and $f$ be the joint cdf and joint density of $\pmb{y}$ with parameters $\pmb{\delta}$,
and let $F_i$ and $f_i$ be the marginal cdf and marginal density for the $i^\text{th}$ random variable with parameters $\pmb{\theta}_i$, we have
\begin{align}
    f(\pmb{y}; \pmb{\theta}_1, \pmb{\theta}_2,\dots \pmb{\theta}_i, \pmb{\delta}) =
    c\{F_1(\pmb{y}_1;\pmb{\theta}_1), F_2(\pmb{y}_2; \pmb{\theta}_2), \dots, F_i(\pmb{y}_1;\pmb{\theta}_i); \pmb{\delta}\}
    \prod^i_{j=1}f_i(\pmb{y}_j;\pmb{\theta}_j)
    \end{align}

For a sample of size $n$, the log-likelihood of functions of the $i^\text{th}$ univariate margin is
\begin{align}
    L_i(\theta_i) = \sum^n_{m=1} \log f_i(y_{mi}; \theta_i),
    \end{align}

and the log-likelihood function for the joint distribution is
\begin{align}
    L(\delta, \theta_1, \theta_2, \dots, \theta_i) = \sum^n_{m=1}\sum^i_{j=1} \log f(y_{mj}; \delta, \theta_1, \theta_2, ..., \theta_i)
    \end{align}

In most cases, one does not have closed form estimators and numerical techniques are needed.
Numerical ML estimation difficulty increase when the total number of parameters increases.
The two-stage estimation is designed to overcome this problem.

The two-stage procedure is
\begin{enumerate}
    \item estimate the univariate parameters from separate univariate likelihoods to get $\tilde{\pmb{\theta}_1}, ..., \tilde{\pmb{\theta}_i}$
    \item maximize $L(\pmb{\delta}, \tilde{\pmb{\theta}_1}, \dots, \tilde{\pmb{\theta}_i})$ over $\pmb{\delta}$ to get $\tilde{\pmb{\delta}}$
    \end{enumerate}

Under regularity conditions
\footnote{Regularity conditions include
1. $\exists \frac{\partial \log f(x;\theta)}{\partial \theta}, \frac{\partial^2 \log f(x;\theta)}{\partial \theta^2}, \frac{\partial^3 \log f(x;\theta)}{\partial \theta^3}$ for all $x$;
2. $\exists g(x), h(x) and H(x)$ such that for $\theta$ in a neighborhood $N(\theta_0)$ the relations
$\left|\frac{\partial f(x;\theta)}{\partial theta}\right| \leq g(x)$,
$\left|\frac{\partial^2 f(x;\theta)}{\partial \theta^2}\right| \leq h(x)$,
$\left|\frac{\partial^3 f(x;\theta)}{\partial \theta^3}\right| \leq H(x)$ hold for all $x$, and
$\int g(x) dx < \infty$, $\int h(x) dx < \infty$, $\mathbb{E}_\theta \{H(X)\} < \infty$ for $\theta \in N(\theta_0)$;
3. For each $\theta \in \Theta$, $0< \mathbb{E}_\theta \left\{
\left(
\frac{\partial \log f(X;\theta)}{\partial \theta}
\right)^2
\right\}$. For detail see section 4.2.2 of~\cite{serfling2009approximation}}
, $(\pmb{\tilde{\theta}}_1,\dots \pmb{\tilde{\theta}}_i, \pmb{\tilde{\delta}})$ is the solution of
\begin{align}
    (\partial L_1 / \partial \pmb{\theta}^\intercal_1,
    \dots, \partial L_i / \partial \pmb{\theta}^\intercal_i, \partial L / \partial \pmb{\pmb{\delta}}^\intercal_1) = \pmb{0}
    \end{align}

For comparison, if we optimize $L$ directly without the two-stage procedure (i.e.~MLE), we solve for
\begin{align}
    (\partial L / \partial \pmb{\theta}^\intercal_1,
    \dots, \partial L / \partial \pmb{\theta}^\intercal_i, \partial L / \partial \pmb{\pmb{\delta}}^\intercal_1) = \pmb{0}
    \end{align}

We denote the two solutions as
$\tilde{\pmb{\eta}} = (\pmb{\tilde{\theta}}_1,\dots \pmb{\tilde{\theta}}_i, \pmb{\tilde{\delta}})$ for two-stage procedure;
$\hat{\pmb{\eta}} =(\pmb{\hat{\theta}}_1,\dots \pmb{\hat{\theta}}_i, \pmb{\hat{\delta}})$ for MLE procedure.
and compare the asymptotic relative efficiency of $\tilde{\pmb{\eta}}$ and $\hat{\pmb{\eta}}$.

Asymptotics: yet to be done.\\
~\cite{kim2007comparison} show the estimation of $\pmb{\theta}$ may be seriously affected.
They compare the two-stage approach and Canonical Maximum Likelihood Method by simulation and
conclude that Canonical Maximum Likelihood is prefered from a computational statistics and data analysis point of view.

\subsection{Canonical Maximum Likelihood Method}\label{subsec:canonical-maximum-likelihood-method}
This approach was studied by~\cite{genest1995semiparametric} and~\cite{shih1995inferences}.
One estimates the margins using empirical CDF
\begin{align}F_X(x)=\frac{1}{n+1}\sum_{i=1}^n 1(X_i \leq x)\end{align},

we maximize the log-likelihood
\begin{align}
    L(\theta) = \sum_{i=1}^n \log [c_\theta \{F_X(X_i), F_Y(Y_i)\}]
    \end{align}

This procedure does not require specification of marginals.





%also by Wang and Ding, 2000; Tsukahara, 2005
%This is also known as pseudo maximum likelihood (PML) and as canonical maximum likelihood (see Cherubini et al., 2004)
%
%Genest and Werker (2002) obtained conditions under which the PMLE is asymptotically efficient.
%
%
