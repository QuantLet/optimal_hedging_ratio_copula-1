%! Author = francis
%! Date = 30.10.20

\section{$D_1$ Operator}
We give a explicit equation of the $D_1$ operator for Archimedean copulae.

The $D_1$ operator is defined as the partial derivatives of the first input to the copula function,
so we fix the second argument while taking derivative with respect to the first, and then evaluate the function.
we have

\begin{align}
\left.\frac{\partial C\{v, g(w)\}}{\partial v} \right\vert_{v=w}
&=
\left.\frac{\partial  \phi^{-1}\{\phi(v)+\phi(g(w))\}}
{\partial  \{\phi(v)+\phi(g(w))\}}
\frac{\partial  [\phi(v)+\phi(g(w))]}
{\partial  v}
\right\vert_{v=w}\\
&=
\left.\frac{\partial   \phi^{-1}\{\phi(v)+\phi(g(w))\}}
{\partial \{\phi(v)+\phi(g(w))\}}
\frac{\partial  \phi(v)}{\partial  v}
\right\vert_{v=w}\\
&=
\frac{\partial   \phi^{-1}\{\phi(w)+\phi(g(w))\}}
{\partial \{\phi(w)+\phi(g(w))\}}
\frac{\partial  \phi(w)}{\partial  w}\\
&,
\text{where } g(w) = F_{R^F}\left[\frac{F^{-1}_{R^S}(w)-r^h}{h}\right]\\
\end{align}

\begin{table}
    \center
    \begin{tabular}{c c c c c}
        Function & Gumbel & Frank & Clayton & Independence\\
        $\phi(t)$    &
        $\{-\log(t)\}^\theta$ &
        $-\ln \left\{
        \frac{\exp(-\theta t)-1}
        {\exp(-\theta)-1}
        \right\}$&
        $\frac{1}{\theta}
        (t^{-\theta}-1)$
        & Same to Gumbel where \theta=1\\
        $\phi^{-1}(t)$ &
        $\exp(-t^{1/\theta})$ &
        $\frac{-1}{\theta}
        \ln[1+ \exp(-t)\{\exp(-\theta)-1\}]$ &
        $(1+\theta t)^{-\frac{1}{\theta}}$
        & \\

        $\partial \phi(t)/\partial t$ &
        $\theta \frac{\phi(t)}{t\log(t)}$ &
        $\frac{\theta \exp(-\theta t)}{\exp(-\theta t)-1}$ &
        $-t^{-(\theta + 1)}$&
        \\
        $\partial \phi^{-1}(t)/\partial t$ &
        $\frac{-1}{\theta}t^{\frac{1}{\theta}-1}\phi^{-1}(t)$&
        $\frac{\exp(-t)}{\theta \{1+\exp(-t)\}}$&
        $\theta (1+\theta t)^{-\frac{1}{\theta}-1}$&
        \end{tabular}\label{tab:archcopula}
\end{table}






