%! Author = francis
%! Date = 30.10.20

\section{$D_1$ Operator}

The $D_1$ operator is given as
\begin{equation*}
    D_1 C_{X,Y}(F_X(x), F_Y(y)) = \p(Y\leq y|X=x).
\end{equation*}
In the context of the above notation, we obtain
\begin{align*}
  D_1 C_{R^s, R^F}\{w, g(w)\} &= \p[R_F\leq F_{F}^{(-1)}\{g(w)\}|
  R_s=F_S^{(-1)}(w)] %
  = \p\{V\leq g(w)| U=w\}\\ %
  &= \frac{\p\{U\in \dd w, V\leq g(w)\}} {\p(U\in \dd w)} %
  = \p\{U\in \dd w| V\leq g(w)\} %
  = \int_0^{g(w)} c(w,v)\, \dd v. 
\end{align*}
The last line can also be written as
\begin{equation*}
  \frac{\partial }{\partial w} C\{w, g(w')\} \big|_{w'=w}.
\end{equation*}



We give an explicit equation of the $D_1$ operator for Archimedean copulae.

The $D_1$ operator is defined as the partial derivatives of the first input to the copula function,
so we fix the second argument while taking derivative with respect to the first, and then evaluate the function.
we have

\begin{align}
\left.\frac{\partial C\{v, g(w)\}}{\partial v} \right\vert_{v=w}
&=
\left.\frac{\partial  \phi^{-1}[\phi(v)+\phi\{g(w)\}]}
{\partial  [\phi(v)+\phi\{g(w)\}]}
\frac{\partial  [\phi(v)+\phi\{g(w)\}]}
{\partial  v}
\right\vert_{v=w}\\
&=
\left.\frac{\partial   \phi^{-1}[\phi(v)+\phi\{g(w)\}]}
{\partial [\phi(v)+\phi\{g(w)\}]}
\frac{\partial  \phi(v)}{\partial  v}
\right\vert_{v=w}\\
&=
\frac{\partial \phi^{-1}[\phi(w)+\phi\{g(w)\}]}
{\partial [\phi(w)+\phi\{g(w)\}]}
\frac{\partial  \phi(w)}{\partial  w}\\
&,
\text{where } g(w) = F_{R^F}\left\{\frac{F^{-1}_{R^S}(w)-r^h}{h}\right\}\\
\end{align}

\begin{table}
    \center
    \begin{tabular}{c | c c c c}
        Function & Gumbel & Frank & Clayton & Independence\\        \hline
        $\phi(t)$    &
        $\{-\log(t)\}^\theta$ &
        $-\ln \left\{
        \frac{\exp(-\theta t)-1}
        {\exp(-\theta)-1}
        \right\}$&
        $\frac{1}{\theta}
        (t^{-\theta}-1)$
        & Same to Gumbel where $\theta=1$\\
        $\phi^{-1}(t)$ &
        $\exp(-t^{1/\theta})$ &
        $\frac{-1}{\theta}
        \log[1+ \exp(-t)\{\exp(-\theta)-1\}]$ &
        $(1+\theta t)^{-\frac{1}{\theta}}$
        & \\

        $\partial \phi(t)/\partial t$ &
        $\theta \frac{\phi(t)}{t\log(t)}$ &
        $\frac{\theta \exp(-\theta t)} {\exp(-\theta t)-1}$ &
        $-t^{-(\theta + 1)}$&
        \\
        $\partial \phi^{-1}(t)/\partial t$ &
        $\frac{-1}{\theta}t^{\frac{1}{\theta}-1}\phi^{-1}(t)$&
        $\frac{1}{\theta}\frac{\exp(-t)\{\exp(-\theta)-1\}}{1+\exp(-t)\{\exp(-\theta)-1\}}$&
        $\theta (1+\theta t)^{-\frac{1}{\theta}-1}$&
        \end{tabular}\caption{Archemdean Copulae's Generator, Generator Inverse, and their derivative.}\label{tab:ArcGenerator}
\end{table}






