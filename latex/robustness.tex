\section{Robustness}\label{sec:robustness}
The study of robustness concerns the stability of statistical estimation procedure under the existence of outliers.
This has an economical meaning to our hedging exercise: Do we want the optimal hedge ratio react to extreme market changes?
In practice, outliers of returns can come from anywhere, for example, a tweet from Elon Musk, a sudden large order from
institutional investor, or an incident of system failure in cryptocurrency exchanges.
Rapid and drastic changes in portfolio weight causes problem of slippage and transaction cost.
Investors should be aware of the cost brought by the sensitivity of the optimal hedge ratio procedure.
\medskip

The discussion of sensitivity or robustness day back to Huber (1981, ..., ...).
Hampel et al 1986 suggest an infinitesimal approach to investigate sensitivity of statistical procedures.
There are three central concepts in this approach, qualitative robustness, influence function, and break-down point.
They are loosely related to the concept of continuity, first derivative of functional, and the distance of a functional to its nearest pole (singularity).
While the first concept is a qualitative feature of a functional, the second the third concepts are practical tools to measure sensitivity quantitatively.
We consider a finite sample case of the second and third concepts.
Details of robustness of risk measures can be found in Cont. \medskip

With a probability space $(\Omega, \mathscr{F}, \bm{H})$,
we denote $M: \Omega \mapsto \mathscr{C}, M \in \{\text{MLE}, \text{MM}, \text{Empirical}\}$ be estimators of interest for distribution of returns,
$\mathscr{C} = \{\text{Gaussian-Copula}, ..., \text{Plackett-Copula}, \text{Empirical-Copula}\} \in \bm{H}$ be a set of bivariate distributions of interest,
$\rho_{h}: \mathscr{C} \mapsto \mathbb{R}$ be a risk measure on the hedged portfolio given $h$,
and finally, $\hat h_\rho = \argmin_h \rho_{h} \circ M$ be a functional to obtain the optimal hedge ratio (OHR) depending on risk measure
$\rho$. \medskip

The influence function of $\hat h_\rho$ with finite sample size $n$ is
\begin{align}
    \text{IF}(\bm{z}; \hat h_\rho) = \frac{\hat h_\rho(\bm{X}_1,...,\bm{X}_n, \bm{z})-
    \hat h_\rho(\bm{X}_1,...,\bm{X}_n)}{\frac{1}{n+1}}
    \end{align}

The equation describes the effect of a single contamination at point $\bm{z}$ on the estimate of OHR,
standardised by the mass of the contamination. \medskip
